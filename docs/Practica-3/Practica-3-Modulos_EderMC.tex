\documentclass[a4paper,12pt]{report}
\usepackage[spanish]{babel}
\usepackage[T1]{fontenc}
\usepackage[utf8]{inputenc}
\usepackage{lmodern}
\usepackage{float}
\usepackage[margin=2.5cm]{geometry}
% ===== Numeración de secciones =====
\renewcommand\thesection{\arabic{section}}
\makeatletter
\makeatother
\usepackage[hidelinks]{hyperref}
% ===== Imagenes =====
\usepackage{graphicx}
\DeclareGraphicsExtensions{.png}
\graphicspath{{images/}}
% ===== Tablas =====
\usepackage{array}
\usepackage{tabularx}
\newcolumntype{Y}{>{\raggedright\arraybackslash}X}
% ===== Estilos para el recuadro bash para API =====
\usepackage{xcolor}
\usepackage{listings}
\definecolor{codebg}{RGB}{245,245,245}
\lstdefinestyle{bash}{
  language=bash,
  basicstyle=\ttfamily\small,
  backgroundcolor=\color{codebg},
  frame=single,
  breaklines=true,
  showstringspaces=false
}

\title{Práctica 2: Odoo como ERP
}
\author{Eder Martínez Castro}
\date{\today}

\begin{document}
\maketitle
\tableofcontents
\clearpage


\section{Objetivo}
\noindent El objetivo de esta práctica es conseguir implantar y configurar en nuestro Odoo los siguientes módulos
\textbf{"Hola mundo"} y \textbf{"Lista de tareas"}, y para finalizar mejoraremos el módulo de \textbf{"Lista de tareas"}
en el modelo y la vista.

\vspace{1em}
\noindent Con esta práctica se busca tener el primer contacto con la creación de módulos en Odoo para poder sacarle mucho
más provecho y personalización a nuestro Odoo.



\section{Activación del modo desarrollador}
\noindent Para poder activar el modo desarrollador en nuestro Odoo lo que tendremos que hacer es lo siguiente, tendremos que hacer el
siguiente comando \texttt{"docker compose up -d"} si no tenemos nuestro contenedor corriendo, cuando ya lo tengamos levantado
lo que haremos será irnos a nuestro navegador favorito y poner la ruta en la que tenemos desplegado nuestro proyecto en mi caso
\texttt{http://localhost:8069}.

\begin{figure}[H]
  \centering
  \includegraphics[width=.20\linewidth]{modo-desarrollador/ir-ajustes.png}
  \caption{Como ir a ajustes de Odoo}
  \label{fig:interfaz}
\end{figure}

\noindent Ya cuando estemos dentro de nuestro Odoo, nos dirigiremos a donde está el icono cuadrado en la cabecera morada, al hacer clic en este icono
se nos abrirá una lista de opciones y seleccionaremos ajustes.

\begin{figure}[H]
    \centering
    \includegraphics[width=.85\linewidth]{modo-desarrollador/developer-mode.png}
    \caption{Modo desarrollador sin activar}
    \label{fig:interfaz}
\end{figure}

\noindent Ahora que ya estamos en ajustes, nos situaremos en la sección de \textbf{opciones generales} y haremos scroll hasta encontrar la
sección de \textbf{herramientas de desarrollador} en clicaremos en la opción de \texttt{activar modo de desarrollador} y se nos actualizará automáticamente.

\begin{figure}[H]
  \centering
  \includegraphics[width=.85\linewidth]{modo-desarrollador/active-developer-mode.png}
  \caption{Modo desarrollador ya activado}
  \label{fig:interfaz}
\end{figure}

\noindent Como podemos ver en esta captura podemos ver que ya tenemos el modo desarrollador activo y que tenemos la opción de
desactivarlo si quisiéramos.

\begin{figure}[H]
  \centering
  \includegraphics[width=.85\linewidth]{modo-desarrollador/cambios-developer.png}
  \caption{Cambios al estar en el modo desarrollador}
  \label{fig:interfaz}
\end{figure}

\noindent Cuando tenemos el modo desarrollador activado el cambio más notable ya que se aprecia que ya no nos
sale la descripción del módulo, si no que nos sale el como se llama la carpeta que aloja el módulo.



\section{Implementación del Módulo (Hola mundo)}
\noindent Ahora nos dirigimos a nuestro \textbf{IDE}, y abriremos nuestro proyecto donde tenemos nuestro \textbf{Odoo dockerizado}, ahora vamos a ver
como crear nuestro primer módulo de una manera muy simple, ya que para este solo necesitaremos \textbf{crear estos 2 archivos}.

\begin{figure}[H]
  \centering
  \includegraphics[width=.45\linewidth]{modulo-hola-mundo/estructura-modulo1.png}
  \caption{Así quedaría la estructura mínima para crear nuestro módulo hola\_mundo}
  \label{fig:interfaz}
\end{figure}

\clearpage
\noindent Explicación de los \textbf{archivos del módulo hola\_mundo}:
\begin{itemize}
  \item \textbf{\_\_init\_\_.py}: En este archivo se indica que nuestro módulo se puede cargar en Odoo.
  \item \textbf{\_\_manifest\_\_.py}: En este archivo tenemos la información básica de nuestro módulo para que Odoo la reconozca.
\end{itemize}

\begin{figure}[H]
  \centering
  \includegraphics[width=.65\linewidth]{modulo-hola-mundo/manifest-module1.png}
  \caption{Contenido del archivo de \_\_manifest\_\_.py}
  \label{fig:interfaz}
\end{figure}

\noindent Solo necesitamos escribir una línea de código para poder mostrar nuestro módulo en Odoo obviamente no tendrá funcionalidad ninguna
pero nos sirve para mostrar lo  mínimo que se necesita para crear un módulo y que se muestre en las aplicaciones de Odoo.

\begin{figure}[H]
  \centering
  \includegraphics[width=1\linewidth]{modulo-hola-mundo/barra-para-actualizar-lista.png}
  \caption{Seleccionar actualizar lista de apliciones en el menú}
  \label{fig:interfaz}
\end{figure}

\noindent Para actualizar la lista de las aplicaciones de nuestro Odoo, iremos al menú y seleccionaremos la opción de \textbf{actualizar lista de aplicaciones}
que como podemos apreciar es la tercera opción de nuestro menú.

\begin{figure}[H]
  \centering
  \includegraphics[width=.65\linewidth]{modulo-hola-mundo/actualizar-lista.png}
  \caption{Actualizar la lista de apliciones para ver el nuevo módulo}
  \label{fig:interfaz}
\end{figure}

\noindent Al clicar en \textbf{actualizar lista de aplicaciones} nos aparecerá esta ventana emergente, donde seleccionaremos el botón que pone \textbf{Actualizar}.

\begin{figure}[H]
  \centering
  \includegraphics[width=.95\linewidth]{modulo-hola-mundo/modulo-creado.png}
  \caption{Módulo Ejemplo01-Hola mundo creado correctamente}
  \label{fig:interfaz}
\end{figure}

\noindent Después de haber actualizado, nos dirigiremos al buscador donde buscaremos el módulo \textbf{hola\_mundo}, y nos debería salir el módulo que acabamos de crear
como se puede apreciar en esta imagen.



\section{Implementación de Módulo (Lista de tareas)}
\noindent Ahora vamos a empezar con la \textbf{creación de un módulo pero este con funcionalidad} donde veremos los pasos para
poder crear un módulo sobre una lista de tareas sencilla, para gestionar las tareas que tengamos que hacer,


\subsection{Crear la estructura de un módulo}
\begin{figure}[H]
  \centering
  \includegraphics[width=1\linewidth]{modulo-lista-tareas/crear_lista_tareas.png}
  \caption{Comando para crear el módulo lista de tareas de Odoo}
  \label{fig:interfaz}
\end{figure}

\noindent Para poder crear nuestro módulo emplearemos el comando \textbf{scaffold} y para ello como estamos trabajando con Odoo desde \textbf{Docker} lo 
primero que deberemos hacer es ejecutar los siguientes comandos.

\vspace{1em}
\noindent Comando para entrar en nuestro contenedor de docker
\begin{lstlisting}[style=bash]
  docker exec -it <CONTAINER> /bin/bash
\end{lstlisting}
  
\vspace{1em}
\noindent Comando para crear la estructura básica de un nuevo módulo en Odoo
\begin{lstlisting}[style=bash]
  odoo scaffold <MODULO_NAME> /mnt/extra-addons
\end{lstlisting}

\begin{figure}[H]
  \centering
  \includegraphics[width=1\linewidth]{modulo-lista-tareas/permisos-lista_tareas.png}
  \caption{Darle permisos al módulo}
  \label{fig:interfaz}
\end{figure}

\noindent Le vamos a \textbf{dar permisos a nuestro módulo} para poder trabajar más cómodamente con él, como por ejemplo permitirnos guardar los cambios que le vayamos
haciendo a nuestro módulo.

\clearpage
\noindent Comando darle permisos a nuestro módulo
\begin{lstlisting}[style=bash]
  chmod 777 -R /mnt/extra-addons/<MODULO-NAME>
\end{lstlisting}

\begin{figure}[H]
  \centering
  \includegraphics[width=.45\linewidth]{modulo-lista-tareas/comprobacion_lista_tareas.png}
  \caption{Comprobación de la estructura de nuestro módulo de Odoo creado mediante el comando \textbf{scaffold}.}
  \label{fig:interfaz}
\end{figure}


\subsection{Lógica de nuestro módulo}
\begin{figure}[H]
  \centering
  \includegraphics[width=1\linewidth]{modulo-lista-tareas/manifest.png}
  \caption{Archivo \_\_manifest\_\_.xml}
  \label{fig:interfaz}
\end{figure}

\noindent Ahora pegaremos el archivo \texttt{\_\_manifest\_\_.xml} este archivo es que \textbf{define y registra tu módulo} de nuestro módulo en Odoo, aquí es
importante el \texttt{'application': True} que nos permitira que nuestro mádulo salga en el filtro de aplicaciones además de otras configuraciones extra.

\begin{figure}[H]
  \centering
  \includegraphics[width=1\linewidth]{modulo-lista-tareas/views.png}
  \caption{Archivo views.xml}
  \label{fig:interfaz}
\end{figure}

\noindent Ahora nos toca copiar el archivo \texttt{views.xml} que nos pasarón de ejemplo pero esto es para odoo:17 y tenemos que modificar un pequeño detalle
ya no se utiliza \textbf{tree} para odoo:18 ahora se emplea \textbf{list}, con esta pequeña modificación ya lo arreglamos. Este archivo es el encargado de
\textbf{configurar la interfaz} de nuestro módulo.

\begin{figure}[H]
  \centering
  \includegraphics[width=1\linewidth]{modulo-lista-tareas/models.png}
  \caption{Archivo models.py}
  \label{fig:interfaz}
\end{figure}

\noindent Por último pegaremos el archivo \texttt{models.xml} este archivo es el encargado de definir el \textbf{modelo de datos} de nuestro módulo de Odoo.

\begin{figure}[H]
  \centering
  \includegraphics[width=1\linewidth]{modulo-lista-tareas/reiniciamos-odoo.png}
  \caption{Reiniciar odoo para aplicar los cambios}
  \label{fig:interfaz}
\end{figure}

\noindent Cuando ya hayamos modificado los dos archivos anteriores, lo que vamos a hacer es 
ejecutar el siguiente comando para reiniciar nuestro contenedor y asegurarnos de tenerlo actualizado
y así evitarnos posibles problemas.

\vspace{1em}
\noindent Comando para reiniciar nuestro contenedor y aplicar los cambios
\begin{lstlisting}[style=bash]
  docker restart <CONTAINER>
\end{lstlisting}


\subsection{Comprobación del módulo creado}
\begin{figure}[H]
  \centering
  \includegraphics[width=1\linewidth]{modulo-lista-tareas/comprobacion-en-odoo-lista_tareas.png}
  \caption{Módulo lista\_tareas creado correctamente}
  \label{fig:interfaz}
\end{figure}

\noindent Después de haber modificado los archivos anteriores y reiniciado nuestro contenedor lo que vamos a hacer 
es lo siguiente, nos dirigiremos a las aplicaciones de Odoo y buscaremos el módulo que acabmos de crear, así que buscaremos
\textbf{lista\_tareas} y procederemos a activar esta aplicación.

\begin{figure}[H]
  \centering
  \includegraphics[width=1\linewidth]{modulo-lista-tareas/modulo-activo.png}
  \caption{Módulo ya activado}
  \label{fig:interfaz}
\end{figure}

\noindent Como podemos ver ya tenemos nuestro módulo \textbf{lista\_tareas} ya activo ya listo para usarlo.

\begin{figure}[H]
  \centering
  \includegraphics[width=.3\linewidth]{modulo-lista-tareas/ir-al-modulo.png}
  \caption{Entraremos en la aplicación de listado de tareas}
  \label{fig:interfaz}
\end{figure}

\noindent Como ya tenemos activo nuestro módulo ahora nos dirigiremos al icono de aplicaciones y 
seleccionaremos nuestra aplicación \textbf{Listado de tareas} y pasaremos al siguiente apartado donde probaremos nuestro módulo.


\subsection{Funcionamiento del módulo}
\begin{figure}[H]
  \centering
  \includegraphics[width=1\linewidth]{modulo-lista-tareas/comprobar-funcionamiento-1.png}
  \caption{Nuestro módulo \textbf{listado de tareas} sin ninguna tarea porque todavía no añadimos ninguna todavía.}
  \label{fig:interfaz}
\end{figure}

\noindent Al entrar en nuestra aplicación \textbf{Listado de tareas} podemos ver que es una interfaz bastante simple y sin color, que cuenta con
los siguientes campos para nuestras tareas: tarea, prioridad, urgente y realizada.

\begin{figure}[H]
  \centering
  \includegraphics[width=1\linewidth]{modulo-lista-tareas/comprobar-funcionamiento-2.png}
  \caption{Agregamos una tarea a nuestro módulo \textbf{listado de tareas} en este caso una no urgente.}
  \label{fig:interfaz}
\end{figure}

\noindent Al crear una nueva tarea \textbf{dependiendo del valor} ingresado en el campo prioridad \textbf{se marcará como urgente o no}, en este caso le pusimos
de valor 8 y como no es >10 pues no se marca como urgente, luego también podemos clicar el checkbox para marcarla como completada o no.

\begin{figure}[H]
  \centering
  \includegraphics[width=1\linewidth]{modulo-lista-tareas/comprobar-funcionamiento-3.png}
  \caption{Agregamos otra tarea a nuestro módulo \textbf{listado de tareas} como prueba adicional.}
  \label{fig:interfaz}
\end{figure}

\noindent En esta nueva tarea podemos ver como le di el valor de 20 y \textbf{ahora nos sale marcado el valor de urgente}, además para probar también la marqué como
realizada para mostrar como se vería.

\vspace{1em}
\noindent Como se puede apreciar \textbf{nuestro módulo cumple su función y funciona correctamente}, pero a continuación lo modificaremos para que por lo menos
se vea más bonito y le agregaremos funcionalidad que realmente pueden ser muy útiles.



\section{Mejoras en el Módulo de (Listea de tareas)}

\begin{figure}[H]
  \centering
  \includegraphics[width=.9\linewidth]{modificacion-lista-tareas/views.png}
  \caption{Modificación del archivo views.xml}
  \label{fig:interfaz}
\end{figure}

\noindent Ahora vamos a explicar los cambios realizados en \texttt{models.py}, agregamos los \textbf{nuevos campos} para: descripción, fecha\_limite,
prioridad\_nivel y estado. Y se agregaron \textbf{badges} para darle colores a prioridad\_nivel y estado.

\begin{figure}[H]
  \centering
  \includegraphics[width=.9\linewidth]{modificacion-lista-tareas/models.png}
  \caption{Modificación del archivo models.py}
  \label{fig:interfaz}
\end{figure}

\noindent Ahora vamos a explicar los cambios realizados en \texttt{models.py}, lo primero fue agregar los \textbf{nuevos campos}: descripción, fecha\_limite,
prioridad\_nivel y estado. \textbf{Se eliminó el campo urgente} y se reemplazó  con el campo prioridad que ahora detecta tareas de prioridad: baja, media y urgente.

\begin{figure}[H]
  \centering
  \includegraphics[width=1\linewidth]{modificacion-lista-tareas/actualizar-modulo.png}
  \caption{Actualizar nuestro módulo}
  \label{fig:interfaz}
\end{figure}

\noindent Para poder obtener los cambios que hicimos para nuestro módulo, buscaremos nuestro \textbf{módulo lista\_tareas} y haremos click
en los tres puntos y seleccionaremos la opción de \textbf{actualizar}. Si esto no funcionará te recomiendo que vuelvas a reiniciar tu contenedor
con el comando: \texttt{docker restart <CONTAINER>}.

\begin{figure}[H]
  \centering
  \includegraphics[width=1\linewidth]{modificacion-lista-tareas/resultado.png}
  \caption{Resultado de nuestras modificaciones de nuestro módulo de Odoo}
  \label{fig:interfaz}
\end{figure}

\noindent Así nos quedó nuestro módulo ya modificado, donde podemos ver que nos quedó con estos campos para gestionar nuestras tareas:
\begin{itemize}
  \item \textbf{Tarea} para el nombre de la tarea
  \item \textbf{Descripción} de nuestra tarea
  \item \textbf{Fecha Límite} de la tareas
  \item La \textbf{prioridad} en la que debemos hacer esta tarea (Baja, Media y Urgente)
  \item Y el \textbf{estado} en la que se encuentra la tarea (Pendiente, En curso y Realizada)
\end{itemize}


\section{Conclusión}
\noindent En esta práctica hemos \textbf{aprendido} a cómo \textbf{crear y configurar nuestros módulos en Odoo desde cero}, empezando desde un muy simple a otro en el que
usamos \texttt{\_\_manifest\_\_.py}, \texttt{models.py} y \texttt{views.xml} para crear un módulo funcional de gestion de tareas.


\end{document}