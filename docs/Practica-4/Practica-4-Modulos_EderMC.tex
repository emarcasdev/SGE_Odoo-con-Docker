\documentclass[a4paper,12pt]{report}
\usepackage[spanish]{babel}
\usepackage[T1]{fontenc}
\usepackage[utf8]{inputenc}
\usepackage{lmodern}
\usepackage{float}
\usepackage[margin=2.5cm]{geometry}
% ===== Numeración de secciones =====
\renewcommand\thesection{\arabic{section}}
\makeatletter
\makeatother
\usepackage[hidelinks]{hyperref}
% ===== Imagenes =====
\usepackage{graphicx}
\DeclareGraphicsExtensions{.png}
\graphicspath{{images/}}
% ===== Tablas =====
\usepackage{array}
\usepackage{tabularx}
\newcolumntype{Y}{>{\raggedright\arraybackslash}X}
% ===== Estilos para el recuadro bash para API =====
\usepackage{xcolor}
\usepackage{listings}
\definecolor{codebg}{RGB}{245,245,245}
\lstdefinestyle{bash}{
  language=bash,
  basicstyle=\ttfamily\small,
  backgroundcolor=\color{codebg},
  frame=single,
  breaklines=true,
  showstringspaces=false
}

\title{Práctica 4: Módulos Odoo Modelo y Vista
}
\author{Eder Martínez Castro}
\date{\today}

\begin{document}
\maketitle
\tableofcontents
\clearpage


\section{Introducción}
\noindent El objetivo de esta práctica es seguir trabajando con el modelo y la vista de los módulos de Odoo, para ello
tendremos que implementar cuatro módulos, \textbf{Lista de tareas}, \textbf{Biblioteca de cómics}, \textbf{Pacientes y médicos}, \textbf{Ciclos formativos}.

\vspace{1em}
\noindent Antes de proceder a empezar con la práctica se nos recomienda revisar los ejemplos del 01 al 06 que tenemos en el siguiente 
repositorio \href{https://github.com/sergarb1/OdooModulosEjemplos.git}{github.com/sergarb1/OdooModulosEjemplos.git}.

\vspace{0.5em}
\noindent Para ello deberemos clonarlo en nuetro equipo utilizando el siguiete comando:
\begin{lstlisting}[style=bash]
  git clone https://github.com/sergarb1/OdooModulosEjemplos.git
\end{lstlisting}

\vspace{0.5em}
\noindent Cuando ya lo tengamos clonado lo que haremos es copiar los 6 ejemplos y pegarlos en nuestro proyecto para poder probarlos y revisarlos.
Y nos debería quedar nuestra carpeta con nuestro módulos tal que así:

\begin{figure}[H]
  \centering
  \includegraphics[width=1\linewidth]{PREVIO/MODULOS-EJEMPLO.png}
  \caption{Carpeta addons, que contiene nuestros módulos de odoo donde podemos ver los módulos de ejemplo del repositorio para probarlos}
  \label{fig:interfaz}
\end{figure}

\noindent Entender estos módulos de ejemplo nos ayudará mucho para poder hacer la actividad 1 de lista de tareas y la 2 de la biblioteca de comics. 

\section{Actividad 01 - Lista de tareas}

\section{Actividad 02 - Biblioteca de cómics}

\section{Actividad 03 - Pacientes y médicos}

\section{Actividad 04 - Ciclos formativos}

\section{Conclusión}

\end{document}