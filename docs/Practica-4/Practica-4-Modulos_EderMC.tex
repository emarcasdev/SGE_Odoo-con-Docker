\documentclass[a4paper,12pt]{report}
\usepackage[spanish]{babel}
\usepackage[T1]{fontenc}
\usepackage[utf8]{inputenc}
\usepackage{lmodern}
\usepackage{float}
\usepackage[margin=2.5cm]{geometry}
% ===== Numeración de secciones =====
\renewcommand\thesection{\arabic{section}}
\makeatletter
\makeatother
\usepackage[hidelinks]{hyperref}
% ===== Imagenes =====
\usepackage{graphicx}
\DeclareGraphicsExtensions{.png}
\graphicspath{{images/}}
% ===== Tablas =====
\usepackage{array}
\usepackage{tabularx}
\newcolumntype{Y}{>{\raggedright\arraybackslash}X}
% ===== Estilos para el recuadro bash para API =====
\usepackage{xcolor}
\usepackage{listings}
\definecolor{codebg}{RGB}{245,245,245}
\lstdefinestyle{bash}{
  language=bash,
  basicstyle=\ttfamily\small,
  backgroundcolor=\color{codebg},
  frame=single,
  breaklines=true,
  showstringspaces=false
}

\title{Práctica 4: Módulos Odoo Modelo y Vista
}
\author{Eder Martínez Castro}
\date{\today}

\begin{document}
\maketitle
\tableofcontents
\clearpage


\section{Introducción}
\noindent El objetivo de esta práctica es seguir trabajando con el modelo y la vista de los módulos de Odoo, para ello
tendremos que implementar cuatro módulos, \textbf{Lista de tareas}, \textbf{Biblioteca de cómics}, \textbf{Pacientes y médicos}, \textbf{Ciclos formativos}.

\vspace{1em}
\noindent Antes de proceder a empezar con la práctica se nos recomienda revisar los ejemplos del 01 al 06 que tenemos en el siguiente 
repositorio \href{https://github.com/sergarb1/OdooModulosEjemplos.git}{github.com/sergarb1/OdooModulosEjemplos.git}.

\vspace{0.5em}
\noindent Para ello deberemos clonarlo en nuetro equipo utilizando el siguiete comando:
\begin{lstlisting}[style=bash]
  git clone https://github.com/sergarb1/OdooModulosEjemplos.git
\end{lstlisting}

\vspace{0.5em}
\noindent Cuando ya lo tengamos clonado lo que haremos es copiar los 6 ejemplos y pegarlos en nuestro proyecto para poder probarlos y revisarlos.
Y nos debería quedar nuestra carpeta con nuestro módulos tal que así:

\begin{figure}[H]
  \centering
  \includegraphics[width=1\linewidth]{PREVIO/MODULOS-EJEMPLO.png}
  \caption{Carpeta addons, que contiene nuestros módulos de Odoo donde podemos ver los módulos de ejemplo del repositorio para probarlos.}
\end{figure}

\noindent Entender estos módulos de ejemplo nos ayudará mucho para poder hacer la actividad 1 de lista de tareas y la 2 de la biblioteca de comics. 

\clearpage{}
\section{Actividad 01 - Lista de tareas}

\subsection{Objetivo}
\noindent Para esta actividad se nos pide coger el código de nuestro módulo de ejemplo 2 de la lista de tareas, para modificarlo e implementar en el que
podamos verlo además del \textbf{modo lista}, poder verlo en \textbf{modo kanban} y \textbf{calendario}.

\subsection{Agregar vista Kanban}
\noindent Para poder implementar la vista estilo kanban solo necesitamos modificar el archivo \texttt{views.xml} que es nuestra vista para poder implementar
este nuevo modo de visualización.

\begin{figure}[H]
  \centering
  \includegraphics[width=0.8\linewidth]{EJERCICIO-1/1.1-AGREGAR-KANBAN.png}
  \caption{Nuevo fragmento de código para mostrar las tareas en formato kanban.}
\end{figure}

\noindent En este fragmento de código implementamos la nueva vista tipo kanban, que nos mostrará nuestras tareas en tarjetas individuales con los datos de la tarea.

\clearpage{}
\noindent Estrucutura principal de la vista Kanban:
\begin{itemize}
    \item \texttt{<kanban>}: muestra la vista en formato kanban como bien indica el nombre
    \item \texttt{<field>}: declaramos los campos que tendra la vista.
    \item \texttt{<templates>}: contiene la plantilla \texttt{kanban-box} donde creamos la tarjeta de la tarea del kanban:
    \begin{itemize}
        \item La tarjeta es interactiva al usar \texttt{oe\_kanban\_global\_click} si hacemos click en ella nos abrira el formulario para poder editarla.
        \item Se muestran los datos de la tarea: nombre, prioridad, urgencia y si está o no realizada.
    \end{itemize}
\end{itemize}

\begin{figure}[H]
  \centering
  \includegraphics[width=0.8\linewidth]{EJERCICIO-1/1.1-AGREGAR-ACTION.png}
  \caption{En el bloque para las acciones de nuestro módulo debemos agregar como se ve en pantalla en el \texttt{view\_mode} el nuevo tipo kanban.}
\end{figure}

\noindent En esta modificación lo que hicimos fue agregar la vista kanban al \texttt{view\_mode}, porque Odoo utiliza este parámetro para determinar que vistas estarán disponibles en la interfaz de nuestro módulo.

\clearpage{}

\subsection{Comprobación de la vista Kanban}
\noindent A continuación vamos a enseñar a través de capturas de pantalla el correcto funcionamiento de nuestra vista kanban.

\begin{figure}[H]
  \centering
  \includegraphics[width=1\linewidth]{EJERCICIO-1/1.1-VISTA-LISTA.png}
  \caption{En esta captura podemos ver la vista modo lista de nuestro módulo de listas de tareas que es la primera que nos saldrá cuando lo abramos.}
\end{figure}

\begin{figure}[H]
  \centering
  \includegraphics[width=0.3\linewidth]{EJERCICIO-1/1.1-CAMBIAR-A-KANBAN.png}
  \caption{Como se ve en esta captura para cambiar de vista nos iremos a estos iconos y pulsaremos en el que pone kanban para poder ver nuestras tareas en este formato}
\end{figure}

\begin{figure}[H]
  \centering
  \includegraphics[width=1\linewidth]{EJERCICIO-1/1.1-VISTA-KANBAN.png}
  \caption{En esta captura podemos ver la vista modo kanban de nuestro módulo de listas de tareas.}
\end{figure}

\noindent Aquí podemos ver nuestras tareas en tarjetas con todos sus campos relevantes, lo que muestra que tenemos la vista del kanban bien implementada.

\subsection{Agregar vista Calendar}
\noindent Para poder implementar la vista estilo calendar deberemos modificar el archivo \texttt{views.xml} que es nuestra vista y el archivo \texttt{models.py} que es nuestro modelo para poder implementar
este nuevo modo de visualización.

\vspace{1em}

\noindent Primero vamos a empezar modificando nuestro \texttt{models.py}:

\begin{figure}[H]
  \centering
  \includegraphics[width=0.8\linewidth]{EJERCICIO-1/1.2-AGREGAR-CAMPOS-AL-MODEL.png}
  \caption{En esta captura podemos ver como hemos agregado a nuestra modelo los campos \texttt{fecha\_inicio} y \texttt{fecha\_fin} para poder marcar el comienzo de una tarea y su final.}
\end{figure}

\noindent Para que nuestra vista calendario pueda mostrar nuestras tareas, necesitamos crear dos nuevos campos para poder marcar la fecha de inicio de la tarea y la fecha de fin de esta.

\begin{itemize}
    \item \texttt{fecha\_inicio}: fecha en la que esta tarea empieza.
    \item \texttt{fecha\_fin}: fecha límite de la tarea.
\end{itemize}

\vspace{1em}

\noindent Ahora vamos a realizar las siguientes modificaciones en nuestro \texttt{views.xml}:

\begin{figure}[H]
  \centering
  \includegraphics[width=0.8\linewidth]{EJERCICIO-1/1.2-AGREGAR-CALENDAR.png}
  \caption{Nuevo fragmento de código para mostrar las tareas en formato calendar.}
\end{figure}

\noindent En este fragmento de código implementamos la nueva vista tipo calendar, que nos mostrará un calendario con nuestras tareas desde el dia que empiezan hasta el dia que deben estar finalizadas.

\vspace{0.5em}

\noindent Estrucutura principal de la vista calendar:
\begin{itemize}
    \item \texttt{<calendar>}: muestra la vista en formato calendario como bien indica el nombre, esta recibe dos atributos:
    \begin{itemize}
        \item \texttt{date\_start="fecha\_inicio"}: indicamos el día de comienzo de la tarea.
        \item \texttt{date\_start="fecha\_fin"}: indicamos el día final de la tarea.
    \end{itemize}
    \item \texttt{<field>}: declaramos los campos que tendra la vista.
\end{itemize}

\begin{figure}[H]
  \centering
  \includegraphics[width=0.8\linewidth]{EJERCICIO-1/1.2-AGREGAR-ACTION.png}
  \caption{En el bloque para las acciones de nuestro módulo debemos agregar como se ve en pantalla en el \texttt{view\_mode} el nuevo tipo calendar.}
\end{figure}

\noindent En esta modificación lo que hicimos fue agregar la vista calendar al \texttt{view\_mode}, como ya explicamos anteriormente esto es porque Odoo utiliza este parámetro para determinar que vistas estarán disponibles en la interfaz de nuestro módulo.

\clearpage{}

\noindent Ahora que ya hemos implementado nuestra vista kanban en el \texttt{views.xml}, pero todavía nos falta hacer unas modificaciones en los bloques
del formulario para poder agregar la fecha de inicio y fin a las tareas cuando las creemos o editemos, además de agregar estos nuevos campos de fecha a
las vistas de lista y kanban:

\begin{figure}[H]
  \centering
  \includegraphics[width=0.8\linewidth]{EJERCICIO-1/1.2-ACTULIZAR-FORM.png}
  \caption{Actualizamos el bloque de formulario para tener un nuevo bloque para agregar las fechas de inicio y fin de la tarea para cuando agregemos una tarea o la editemos.}
\end{figure}

\noindent En el formulario, agregamos un nuevo grupo para las fechas de la tarea donde tenemos los campos de fecha de inicio y fecha final de la tarea, para cuando creemos y editemos una tarea podamos agregar las fechas.

\begin{figure}[H]
  \centering
  \includegraphics[width=0.8\linewidth]{EJERCICIO-1/1.2-ACTUALIZAR-LIST.png}
  \caption{Actualizamos el bloque de la vista list para mostrar los nuevos campos de fechas de inicio y fin de la tarea.}
\end{figure}

\noindent En la vista de lista, agregamos los nuevos campos de fecha de inicio y fecha de final de la tarea como nuevas columnas.

\begin{figure}[H]
  \centering
  \includegraphics[width=0.8\linewidth]{EJERCICIO-1/1.2-ACTUALIZAR-KANBAN.png}
  \caption{Actualizamos el bloque de la vista kanban para mostrar los nuevos campos de fechas de inicio y fin de la tarea.}
\end{figure}

\noindent En la vista de kanban, agregamos los nuevos campos de fecha de inicio y fecha de final de la tarea en la tarjeta del kanban.

\subsection{Comprobación de la vista Calendar}
\noindent A continuación vamos a enseñar a través de capturas de pantalla el correcto funcionamiento de la vista calendar que acabamos de implementar.

\begin{figure}[H]
  \centering
  \includegraphics[width=0.3\linewidth]{EJERCICIO-1/1.2-CAMBIAR-A-CALENDAR.png}
  \caption{Como se pueder ver en esta captura para poder cambiar de vista nos iremos a estos iconos y pulsaremos en el que pone calendario.}
\end{figure}

\begin{figure}[H]
  \centering
  \includegraphics[width=1\linewidth]{EJERCICIO-1/1.2-VISTA-CALENDAR.png}
  \caption{En esta captura podemos ver la vista modo calendar de nuestro módulo de listas de tareas.}
\end{figure}

\noindent Aquí podemos ver como la tarea que creamos de \textbf{realizar un pedido para un cliente}, empieza el jueves 4 y termina el  viernes 18 de diciembre de 2025, lo que muestra que tenemos la vista del calendario bien implementada.

\section{Actividad 02 - Biblioteca de cómics}

\subsection{Objetivo}
\noindent Para esta actividad se nos pide ampliar el módulo del ejemplo 3 de la biblioteca de cómics simple para agregar la gestión de socios
y la gestión de ejemplares de préstamo. Para ello crearemos nuevos modelos que nos permiten almacenar los datos de los socios y de los ejemplares,
así como poder registrar los préstamos indicando el socio, la fecha de inicio y la fecha de devolución, aplicando las restricciones indicadas a las fechas.

\subsection{Modelo Socios}
\noindent A continuación vamos a crear el modelo de socios para nuesto módulo, para ello tendremos que crear la vista y el modelo además de realizar unas modificaciones para emplearlo.

\begin{figure}[H]
  \centering
  \includegraphics[width=0.8\linewidth]{EJERCICIO-2/2.1-AGREGAR-MODELO-SOCIO.png}
  \caption{Captura del archivo \texttt{biblioteca\_socio.py} donde tenemos el modelo para gestionar socios.}
\end{figure}

\noindent Creamos el archivo \texttt{biblioteca\_socio.py} y definimos el modelo "biblioteca.socio", que se encargará de gestionar
los socios de la biblioteca de cómics. Cada socio tiene un identificador único, además de su nombre y apellido.

\vspace{0.5em}

\noindent Los atributos principales del modelo son:

\begin{itemize}
  \item id\_socio: identificador único para cada socio.
  \item nombre: nombre del socio.
  \item apellido: apellido del socio. 
\end{itemize}

\noindent Además, usamos una restricción SQL para asegurar que el \texttt{id\_socio} sea único.

\begin{figure}[H]
  \centering
  \includegraphics[width=0.8\linewidth]{EJERCICIO-2/2.1-IMPORTAR-MODELO.png}
  \caption{Captura de que importamos el nuevo modelo socio en \texttt{models/\_\_init\_\_.py}.}
\end{figure}

\noindent Para que Odoo cargue nuestro modelo de socios al iniciar el módulo, importamos el modelo en el archivo \texttt{models/\_\_init\_\_.py}
como ya teniamos el del modelo de cómic.

\begin{figure}[H]
  \centering
  \includegraphics[width=0.8\linewidth]{EJERCICIO-2/2.1-AGREGAR-VISTA-SOCIO.png}
  \caption{Captura del archivo \texttt{biblioteca\_socio.xml} donde tenemos definida la vista para socios.}
\end{figure}

\noindent Creamos el archivo \texttt{biblioteca\_socio.xml} donde creamos la vista \texttt{socio} para poder gestionar los socios desde una interfaz donde tenemos lo siguiente:

\begin{itemize}
  \item \textbf{Vista formulario:} Esto nos permite tener un formulario donde podremos crear o editar la información de los socios.
  \item \textbf{Vista lista:} Esto nos muestra tados los socios en un listado formato tabla con los campos: identificador, nombre y apellido.
\end{itemize}

\noindent También agregamos la acción y el menú correspondientes para poder acceder al listado de socios desde la interfaz.

\begin{figure}[H]
  \centering
  \includegraphics[width=0.8\linewidth]{EJERCICIO-2/2.1-AGREGAR-VISTA-MANIFEST.png}
  \caption{Captura de que incluimos la nueva vista de socio en \texttt{\_\_manifest\_\_.py}.}
\end{figure}

\noindent Para que Odoo cargue nuestro vista de socios, tenemos que agregarlo en \texttt{data} en el archivo \texttt{\_\_manifest\_\_.py}
justo debajo de la vista de cómics.

\begin{figure}[H]
  \centering
  \includegraphics[width=0.8\linewidth]{EJERCICIO-2/2.1-REGLAS-ACCESO.png}
  \caption{Captura donde actualizamos las reglas de acceso donde asignamos los permisos para socios.}
\end{figure}

\noindent Ahora, definimos las reglas de acceso, permitiendo a todos los usuarios poder ver los socios, pero solo los miembros del \textbf{grupo bibliotecario}
podrán crear, editar y eliminar los socios.


\subsection{Modelo Ejemplares}
\noindent A continuación vamos a crear el modelo de ejemplares para nuesto módulo, para ello tendremos
que crear la vista y el modelo además de realizar unas modificaciones para poder emplearlo.

\begin{figure}[H]
  \centering
  \includegraphics[width=0.8\linewidth]{EJERCICIO-2/2.2-AGREGAR-MODELO-EJEMPLAR.png}
  \caption{Captura del archivo \texttt{biblioteca\_ejemplares.py} donde tenemos el modelo para gestionar los ejemplares.}
\end{figure}

\noindent Creamos el archivo \texttt{biblioteca\_ejemplar.py} y definimos el modelo \texttt{"biblioteca.comic.ejemplar"}. Este modelo nos permite gestionar cada ejemplar de un comic de nuestra biblioteca.
Cada ejemplar tiene un identificador único, una referencia al cómic al que pertenece, un estado y opcionalmente, un socio al que esta prestado y las fechas de prestamo y devolución.

\vspace{0.5em}

\noindent Los atributos principales del modelo son:

\begin{itemize}
  \item id\_ejemplar: identificador único para cada ejemplar.
  \item comic\_name: tenemos una relación \texttt{Many2one} hacia el modelo de cómic, ya que cada ejemplar pertenece a un único cómic.
  \item socio\_id: tenemos una relación \texttt{Many2one} hacia el modelo de socio, ya que cada ejemplar esta prestado a un socio.
  \item estado: indica si el ejemplar está disponible o está prestado.
  \item fecha\_prestamo: fecha en la que se presto el ejemplar.
  \item fecha\_devolucion: fecha prevista en la que se devolvera el ejemplar.
\end{itemize}

\noindent Además, usamos una restricción SQL para asegurar que el \texttt{id\_ejemplar} sea único.

\vspace{1em}

\noindent Agregamos las restricciones para las fechas:

\vspace{0.5em}

\noindent Incluimos dos comprobaciones mediante \texttt{@api.constrains}, así si se intenta guardar un ejemplar que incumpla alguna
de estás reglas que nos salte un error:

\begin{enumerate}
  \item Que la fecha de préstamo no puede ser posterior al día actual.
  \item Que la fecha de devolución no puede ser anterior al día actual.
\end{enumerate}

\begin{figure}[H]
  \centering
  \includegraphics[width=0.8\linewidth]{EJERCICIO-2/2.2-IMPORTAR-MODELO.png}
  \caption{Captura de que importamos el nuevo modelo ejemplar en \texttt{models/\_\_init\_\_.py}.}
\end{figure}

\noindent Para que Odoo cargue nuestro modelo de ejemplar al iniciar el módulo, importamos el modelo en el archivo \texttt{models/\_\_init\_\_.py}
como ya teniamos el del modelo de cómic y socio.

\begin{figure}[H]
  \centering
  \includegraphics[width=0.8\linewidth]{EJERCICIO-2/2.2-AGREGAR-VISTA-EJEMPLAR.png}
  \caption{Captura del archivo \texttt{biblioteca\_ejemplar.xml} donde tenemos definida la vista para los ejemplares.}
\end{figure}

\noindent Creamos el archivo \texttt{biblioteca\_ejemplar.xml} donde creamos la vista \texttt{ejemplar} para poder gestionar los ejemplares desde una interfaz donde tenemos lo siguiente:

\begin{itemize}
  \item \textbf{Vista formulario:} Esto nos permite tener un formulario donde podremos crear o editar la información de los ejemplares.
  \item \textbf{Vista lista:} Esto nos muestra tados los ejemplares en un listado formato tabla con los campos: identificador del ejemplar, nombre del cómic, estado, identificador del socio, fecha del prestamo y de devolución.
\end{itemize}

\noindent También agregamos la acción y el menú correspondientes para poder acceder al listado de ejemplares desde la interfaz.

\begin{figure}[H]
  \centering
  \includegraphics[width=0.8\linewidth]{EJERCICIO-2/2.2-AGREGAR-VISTA-MANIFEST.png}
  \caption{Captura de que incluimos la nueva vista de ejemplar en \texttt{\_\_manifest\_\_.py}.}
\end{figure}

\noindent Para que Odoo cargue nuestro vista de ejemplares, tenemos que agregarlo en \texttt{data} en el archivo \texttt{\_\_manifest\_\_.py}
justo debajo de la vista de socios.

\begin{figure}[H]
  \centering
  \includegraphics[width=0.8\linewidth]{EJERCICIO-2/2.2-REGLAS-ACCESO.png}
  \caption{Captura donde actualizamos las reglas de acceso donde asignamos los permisos para ejemplares.}
\end{figure}

\noindent Ahora, definimos las reglas de acceso, permitiendo a todos los usuarios poder ver los ejemplares, pero solo los miembros del \textbf{grupo bibliotecario}
podrán crear, editar y eliminar los ejemplares.

\subsection{Comprobación de su funcionamiento}

\begin{figure}[H]
  \centering
  \includegraphics[width=1\linewidth]{EJERCICIO-2/2.3-VISTA-COMICS.png}
  \caption{En esta captura tenemos la lista con todos los cómics que creamos para nuestra biblioteca.}
\end{figure}

\noindent Como podemos ver tenemos creados cinco cómics y vemos que se muestra correctamente en el formato lista, ademas podemos ver en el
menú las opciones cómics, socios y ejemplares, para movernos de modelo.

\begin{figure}[H]
  \centering
  \includegraphics[width=1\linewidth]{EJERCICIO-2/2.3-VISTA-LIST-SOCIOS.png}
  \caption{En esta captura tenemos la lista con todos los socios que hemos creamos para nuestra biblioteca.}
\end{figure}

\noindent Como podemos ver estamos en la parte del modulo de socios donde los gestionamos, creamos cinco socios de ejemplos y los mostramos en el formato
lista correctamente.

\begin{figure}[H]
  \centering
  \includegraphics[width=0.8\linewidth]{EJERCICIO-2/2.3-VISTA-FORM-SOCIOS.png}
  \caption{En está captura tenemos el formulario para crear y editar los socios.}
\end{figure}

\noindent Como podemos ver ahora estamos en la pantalla de crear o editar un socio donde podemos ver que tenemos los campos: identificador, nombre y apellido.

\begin{figure}[H]
  \centering
  \includegraphics[width=1\linewidth]{EJERCICIO-2/2.3-VISTA-LIST-EJEMPLARES.png}
  \caption{En esta captura tenemos la lista con todos los ejemplares que hemos creamos para nuestra biblioteca.}
\end{figure}

\noindent Ahora nos encontramos en la parte del modulo de ejemplares donde los gestionamos, he creado 2 ejemplares de ejemplos y los mostramos en el formato
lista correctamente.

\begin{figure}[H]
  \centering
  \includegraphics[width=0.8\linewidth]{EJERCICIO-2/2.3-VISTA-FORM-EJEMPLARES.png}
  \caption{En está captura tenemos el formulario para crear y editar nuestros ejemplares.}
\end{figure}

\noindent Como podemos ver ahora estamos en la pantalla de crear o editar un ejemplar donde podemos ver que tenemos los campos: identificador del ejemplar, nombre del cómic, estado, a que socio está prestado, en que fecha se presto y cuando se va a devolver.

\vspace{1em}

\noindent Por último vamos a mostrar las comprobaciones de que nuestras restricciones en las fechas en ejemplares están funcionando correctamente.

\begin{figure}[H]
  \centering
  \includegraphics[width=1\linewidth]{EJERCICIO-2/2.3-COMPROBACION-RESTRICCION-1.png}
  \caption{En está captura comprobamos que la fecha de préstamo no puede ser posterior al día actual.}
\end{figure}

\noindent Aquí podemos ver como al intentar guardar un nuevo ejemplar con la fecha de prestamo 13/12/2025 nos salta el error ya que el dia en la intentamos registrar el ejemplar era 12/12/2025 y como definimos en la restricción no podemos registrar un inicio de prestamo en un fecha futura.

\begin{figure}[H]
  \centering
  \includegraphics[width=1\linewidth]{EJERCICIO-2/2.3-COMPROBACION-RESTRICCION-2.png}
  \caption{En está captura comprobamos que la fecha prevista de devolución no puede ser anterior al día actual.}
\end{figure}

\noindent Aquí podemos ver como al intentar guardar un nuevo ejemplar con la fecha de devolución 11/12/2025 nos salta el error ya que el dia en la intentamos registrar el ejemplar era 12/12/2025 y como definimos en la restricción no podemos registrar una devolución de prestamo en un fecha ya pasada al día actual.

\section{Actividad 03 - Pacientes y médicos}

\section{Actividad 04 - Ciclos formativos}

\section{Conclusión}

\end{document}